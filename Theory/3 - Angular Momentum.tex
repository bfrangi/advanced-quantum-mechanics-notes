\section{Angular Momentum}

\subsection{Orbital angular momentum}

\subsubsection{Classical orbital angular momentum}

\begin{wrapfigure}{r}{0.5\textwidth}
  \centering
  \includegraphics[width=0.5\textwidth]{images/classical_angular_momentum.png}
  \caption{Classical orbital angular momentum}
  \label{fig:classical_angular_momentum}
\end{wrapfigure}

In classical mechanics, the angular momentum of a particle relative to some axis is defined as:

\begin{equation}
    \vec{L} = \vec{r} \times \vec{p}
\end{equation}

where $\vec{r}$ is the position vector of the particle with respect to a point on the axis of rotation and $\vec{p}$ is its momentum. 

The total angular momentum of a system of particles is the sum of angular momenta of the individual particles:
\begin{equation}
    \vec{L}_\text{total} = \sum_i \vec{r}_i \times \vec{p}_i
\end{equation}

The total angular momentum varies in time according to the net external torque, which we can obtain by differentiating the total angular momentum with respect to time:
\begin{equation}
    \vec{\tau}_\text{total} = \sum_i \vec{\tau}_{\text{ext},i} = \frac{d\vec{L}_\text{total}}{dt}
\end{equation}

It follows that the total angular momentum of a system is conserved if the resultant external torque acting on the system is zero.

\subsubsection{Orbital angular momentum in quantum mechanics}

As discussed in \textbf{Section \ref{observables_and_operators}}, to obtain the quantum mechanical operator for orbital angular momentum from its classical definition, we can apply quantization to the classical expression provided in the previous section:
\begin{equation}
    \vec{L} = \vec{R} \times \vec{P} = -i\hbar \vec{R}\times \vec{\nabla}
\end{equation}

For a system of (spin-less) particles, the total angular momentum is defined as:
\begin{equation}
    \vec{L}_\text{total} = \sum_i \vec{R}_i \times \vec{P}_i
\end{equation}

The different cartesian components of the angular momentum operator are:
\begin{equation}
    \begin{split}
        L_x &= YP_z - ZP_y = -i\hbar \left(Y\frac{\partial}{\partial z} - Z\frac{\partial}{\partial y}\right) \\
        L_y &= ZP_x - XP_z = -i\hbar \left(Z\frac{\partial}{\partial x} - X\frac{\partial}{\partial z}\right) \\
        L_z &= XP_y - YP_x = -i\hbar \left(X\frac{\partial}{\partial y} - Y\frac{\partial}{\partial x}\right)
    \end{split}
\end{equation}
We can also define the square of the angular momentum operator:
\begin{equation}
    L^2 = L_x^2 + L_y^2 + L_z^2
\end{equation}
As expected from operators corresponding to observables, all angular momentum operators are Hermitian.

\subsubsection{Commutation relations}

The commutation relations for the orbital angular momentum operators are:
\begin{equation}
    \begin{split}
        [L_x, L_y] &= L_xL_y - L_yL_x = i\hbar L_z \\
        [L_y, L_z] &= L_yL_z - L_zL_y = i\hbar L_x \\
        [L_z, L_x] &= L_zL_x - L_xL_z = i\hbar L_y
    \end{split}
\end{equation}

\subsubsection{General formalism of angular momentum}

We have just defined the orbital angular momentum. However there exists a more general formalism of angular momentum, which is the \textit{total} angular momentum. Its corresponding operator is $\vec{J}$, defined by its three components that satisfy:
\begin{equation} \label{eq:commutation_relations_angular_momentum}
    \begin{split}
        [J_x, J_y] &= i\hbar J_z \\
        [J_y, J_z] &= i\hbar J_x \\
        [J_z, J_x] &= i\hbar J_y
    \end{split}\qquad \vec{J}\,^2 = J_x^2 + J_y^2 + J_z^2\qquad [\vec{J}\,^2, J_k] = 0\quad (k = x, y, z)
\end{equation}

Thanks to these commutation relations, we know that we cannot measure the three components of the total angular momentum simultaneously. However, we \textit{can} simultaneously measure the total angular momentum squared $J^2$ and one of its components. There is possibility to find simultaneous eigenstates of $J^2$ and any component of $J$. However, we can only choose one component of $J$ to be measured simultaneously with $J^2$. By convention, we choose $J_z$, so that we work with a basis of eigenvectors that is common to $J^2$ and $J_z$ in all our calculations\footnote{Note that this is just a convention. There is \textit{nothing} special about the $z$ direction compared to $x$ and $y$.}.

\textit{Eigenstates of the total angular momentum operators}

Let us now look for the joint eigenstates of $J^2$ and $J_z$ and their corresponding eigenvalues. Denoting the joint eigenstates by $\ket{\alpha, \beta}$, and the corresponding eigenvalues of $J^2$ and $J_z$ by $\hbar^2\alpha$ and $\hbar \beta$, respectively, we have:
\begin{equation} \label{eq:angular_momentum_eigenstates}
    \begin{split}
        J^2\ket{\alpha, \beta} &= \hbar^2\alpha\ket{\alpha, \beta} \\
        J_z\ket{\alpha, \beta} &= \hbar\beta\ket{\alpha, \beta}
    \end{split}
\end{equation}
The factor $\hbar$ is introduced so that $\alpha$ and $\beta$ are dimensionless. We also assume that the eigenstates are orthonormal.

Now we need to introduce \textit{raising} and \textit{lowering} operators $J_+$ and $J_-$, respectively, which are defined as:
\begin{equation}
    J_\pm = J_x \pm iJ_y
\end{equation}
This leads to:
\begin{equation} \label{eq:jx_and_jy_in_terms_of_jp_and_jm}
    J_x = \frac12(J_+ + J_-)\qquad J_y = \frac{1}{2i}(J_+ - J_-)
\end{equation}
hence:
\begin{equation}
    J_x^2 = \frac14(J_+^2+J_+J_-+J_-J_++J_-^2)\qquad J_y^2 = -\frac{1}{4}(J_+^2-J_+J_--J_-J_++J_-^2)
\end{equation}
Using the commutation relations from \textbf{Equation \ref{eq:commutation_relations_angular_momentum}}, we can easily obtain:
\begin{equation} \label{eq:commutation_relations_angular_momentum_2}
    [\vec{J}\,^2, J_\pm] = 0\qquad [J_+, J_-] = 2\hbar J_z \qquad [J_z, J_\pm] = \pm\hbar J_\pm
\end{equation}
And also:
\begin{equation} \label{j_p_and_j_m_products}
    \begin{split}
        J_+J_- = J_x^2 + J_y^2 + \hbar J_z = \vec{J}\,^2 - J_z^2 + \hbar J_z\\
        J_-J_+ = J_x^2 + J_y^2 - \hbar J_z = \vec{J}\,^2 - J_z^2 - \hbar J_z
    \end{split} 
\end{equation}
These relations lead to:
\begin{equation}
    \vec{J}\,^2 = J_\pm J_\mp + J_z^2 \mp \hbar J_z = \frac12(J_+ J_- + J_- J_+) + J_z^2
\end{equation}

Since $J_\pm$ do not commute with $J_z$, the kets $\ket{\alpha, \beta}$ are not eigenstates of $J_\pm$. Using the expressions in \textbf{Equation \ref{eq:commutation_relations_angular_momentum_2}}, we can obtain:
\begin{equation} \label{eq:jz_jpm_eigenstates}
    \begin{split}
        J_z(J_\pm\ket{\alpha, \beta}) &= (J_\pm J_z \pm \hbar J_\pm)\ket{\alpha, \beta} = J_\pm J_z \ket{\alpha, \beta} \pm \hbar J_\pm \ket{\alpha, \beta} = \\
        &= \hbar \beta J_\pm \ket{\alpha, \beta} \pm \hbar J_\pm \ket{\alpha, \beta} = \hbar(\beta \pm 1)(J_\pm\ket{\alpha, \beta})
    \end{split}
\end{equation} 
hence the ket $J_\pm\ket{\alpha, \beta}$ is an eigenstate of $J_z$ with eigenvalue $\hbar(\beta \pm 1)$. Since $\vec{J}\,^2$ commutes with $J_z$, $J_\pm\ket{\alpha, \beta}$ is also an eigenstate of $\vec{J}\,^2$. Using \textbf{Equation \ref{eq:commutation_relations_angular_momentum_2}} again, which tells us that $\vec{J}\,^2$ commutes with $J_\pm$, we can determine the eigenvalue, which is $\hbar^2\alpha$:
\begin{equation} \label{eq:j2_jpm_eigenstates}
    \vec{J}\,^2(J_\pm\ket{\alpha, \beta}) = J_\pm \vec{J}\,^2\ket{\alpha, \beta} = J_\pm \hbar^2\alpha \ket{\alpha, \beta} = \hbar^2\alpha(J_\pm \ket{\alpha, \beta})
\end{equation}

If we rewrite \textbf{Equation \ref{eq:jz_jpm_eigenstates}} and \textbf{Equation \ref{eq:j2_jpm_eigenstates}} in terms of $\ket{\alpha',\beta'} = J_\pm \ket{\alpha,\beta}$:
\begin{equation}
    \begin{split}
        J_z\ket{\alpha', \beta'} &= \hbar(\beta \pm 1)\ket{\alpha', \beta'} \\
        \vec{J}\,^2\ket{\alpha', \beta'} &= \hbar^2\alpha\ket{\alpha', \beta'}
    \end{split}
\end{equation}

and if we compare with \textbf{Equation \ref{eq:angular_momentum_eigenstates}}, we can infer that $\alpha' = \alpha$ and $\beta' = \beta \pm 1$. In other words, the ket $J_\pm\ket{\alpha,\beta}$ is proportional to $\ket{\alpha, \beta\pm 1}$ (any eigenket multiplied by a constant is also an eigenket, although it may not be normalised), and we can write:
\begin{equation} \label{eq:j_p_and_j_m_eigenstates}
    J_\pm\ket{\alpha, \beta} = C_{\alpha\beta}^\pm\ket{\alpha, \beta\pm 1}
\end{equation}

So, when the operators $J_\pm$ act on a ket $\ket{\alpha, \beta}$, they do not change the first quantum number $\alpha$, but they increase (or decrease) the second quantum number $\beta$ by one unit. Hence the names \textit{raising} and \textit{lowering} operators.

Note that, for a given eigenvalue $\alpha$ of $\vec{J}\,^2$, there exists an upper limit for the \textit{absolute value}\footnote{That is to say, $\beta$ is bounded from above \textit{and} below.} of the quantum number $\beta$. This is due to the fact that the operator $\vec{J}\,^2 - J_z^2$ is positive definite, as the matrix elements of $\vec{J}\,^2 - J_z^2 = J_x^2 + J_y^2 \geq 0$ are non-negative, so we can write:
\begin{equation}
    \braket{\alpha, \beta| \vec{J}\,^2 - J_z^2 | \alpha, \beta} = \hbar^2(\alpha - \beta^2) \geq 0\quad \Longrightarrow\quad \alpha \geq\beta^2
\end{equation} 

Since $\beta$ has an upper limit, $\beta_\text{max}$, there must exist a state $\ket{\alpha, \beta_\text{max}}$ which cannot be raised further:
\begin{equation}
    J_+\ket{\alpha, \beta_\text{max}} = 0
\end{equation}

Using this, along with \textbf{Equation \ref{j_p_and_j_m_products}}, we can obtain:
\begin{equation}
    J_-J_+\ket{\alpha,\beta_\text{max}} = (\vec{J}\,^2 - J_z^2 - \hbar J_z)\ket{\alpha,\beta_\text{max}} = \hbar^2(\alpha-\beta_\text{max}^2 - \beta_\text{max})\ket{\alpha,\beta_\text{max}} = 0
\end{equation}
hence:
\begin{equation}
    \alpha = \beta_\text{max}(\beta_\text{max} + 1)
\end{equation}

Since $\beta$ has a lower limit, $\beta_\text{min}$, there must exist a state $\ket{\alpha, \beta_\text{min}}$ which cannot be lowered further, which we reach after $n$ successive applications of $J_-$ on $\ket{\alpha, \beta_\text{max}}$:
\begin{equation} \label{eq:alpha_beta_max}
    J_-\ket{\alpha, \beta_\text{min}} = 0
\end{equation}

Using this, along with \textbf{Equation \ref{j_p_and_j_m_products}}, we can obtain:
\begin{equation}
    J_+J_-\ket{\alpha,\beta_\text{min}} = (\vec{J}\,^2 - J_z^2 + \hbar J_z)\ket{\alpha,\beta_\text{min}} = \hbar^2(\alpha-\beta_\text{min}^2 + \beta_\text{min})\ket{\alpha,\beta_\text{min}} = 0
\end{equation}
hence:
\begin{equation} \label{eq:alpha_beta_min}
    \alpha = \beta_\text{min}(\beta_\text{min} - 1)
\end{equation}

Comparing \textbf{Equation \ref{eq:alpha_beta_max}} and \textbf{Equation \ref{eq:alpha_beta_min}}, we can infer that:
\begin{equation}
    \beta_\text{max} = -\beta_\text{min}
\end{equation}

Since $\beta_\text{min}$ was reached after $n$ successive applications of $J_-$ on $\ket{\alpha, \beta_\text{max}}$, it follows that:
\begin{equation}
    \beta_\text{max} = \beta_\text{min} + n, \qquad n\in \N
\end{equation}

Combining the last two equations, we obtain:
\begin{equation}
    \beta_\text{min} = -\frac{n}{2}\qquad \beta_\text{max} = \frac{n}{2}
\end{equation}

Which means that $\beta_\text{max}$ is either an integer or a half-odd-integer. We can now introduce the notation:
\begin{equation}
    j = \beta_\text{max} = \frac{n}{2}\qquad m = \beta
\end{equation}
hence, we can express $\alpha$ as:
\begin{equation}
    \alpha = j(j+1)
\end{equation}m = -j, -j+1, \dots, j-1, j
And we can infer that the values of $m$ lie between $-j$ and $j$:
\begin{equation}
    -j \leq m \leq j
\end{equation}

We can now summarize the results we have obtained so far:
\begin{definition}
    The eigenvalues of $\vec{J}\,^2$ and $J_z$ corresponding to the joint eigenvectors $\ket{j,m}$ are given, respectively, by $\hbar^2j(j+1)$ and $\hbar m$:
    \begin{equation}
        \vec{J}\,^2\ket{j,m} = \hbar^2j(j+1)\ket{j,m}\qquad J_z\ket{j,m} = \hbar m\ket{j,m}
    \end{equation}
    where $j = 0,\, 1/2,\, 1,\, 3/2,\, ...$ and $m = -j, -(j-1), \dots, j-1, j$. We see that the spectra of the angular momentum operators $\vec{J}\,^2$ and $J_z$ are discrete. Since the eigenstates corresponding to different angular momenta are orthogonal, and since the angular momentum spectra are discrete, the orthonormality condition is:
    \begin{equation}
        \braket{j',m'|j,m} = \delta_{j',j}\delta_{m',m}
    \end{equation}
\end{definition}

Let us now determine the normalization constant $C_{\alpha\beta}^\pm$ in \textbf{Equation \ref{eq:j_p_and_j_m_eigenstates}}, which we can rewrite in terms of $j$ and $m$ as:
\begin{equation}
    J_\pm\ket{j,m} = C_{jm}^\pm\ket{j,m\pm 1}
\end{equation}
Since $\ket{j,m+1}$ is normalized:
\begin{equation}
    (J_+\ket{j,m})^\dagger (J_+\ket{j,m}) = |C_{jm}^+|^2\braket{j, m+1|j, m+1} = |C_{jm}^+|^2
\end{equation}

Since $J_+ = J_x + iJ_y$ and the operators $J_x$ and $J_y$ are Hermitian, we have:
\begin{equation}
    J_+^\dagger = (J_x+iJ_y)^\dagger = J_x^\dagger -iJ_y^\dagger = J_x - iJ_y = J_-
\end{equation}
So we can also write:
\begin{equation}
    \braket{j,m|J_-J_+|j,m} = \braket{j,m|J_+^\dagger J_+|j,m} = (J_+\ket{j,m})^\dagger (J_+\ket{j,m}) = |C_{jm}^+|^2
\end{equation}

But since $J_-J_+ = \vec{J}\,^2 - J_z^2 - \hbar J_z$ and $\ket{j,m}$ is orthonormal, we can also write:

\begin{equation}
    \begin{split}
        |C_{jm}^+|^2 &= \braket{j,m|J_-J_+|j,m} = \braket{j,m|\vec{J}\,^2 - J_z^2 - \hbar J_z|j,m} = \\
        &= \braket{j,m|\vec{J}\,^2|j,m} - \braket{j,m|J_z^2|j,m} - \hbar\braket{j,m|J_z|j,m} = \\
        &= \braket{j,m|\hbar^2j(j+1)|j,m} - \braket{j,m|\hbar^2m^2|j,m} - \hbar\braket{j,m|\hbar m|j,m} = \\
        &= \hbar^2j(j+1)\braket{j,m|j,m} - \hbar^2m^2\braket{j,m|j,m} - \hbar^2m\braket{j,m|j,m} = \\
        &= \hbar^2j(j+1) - \hbar^2m^2 - \hbar^2m \\
        &= \hbar^2(j(j+1)-m(m+1))
    \end{split}
\end{equation}

So we conclude that\footnote{Note that here we would actually have to add a phase factor $e^{i\theta}$ so that $C_{jm}^+ = \hbar \sqrt{j(j+1)-m(m+1)}e^{i\theta}$. This is beacuse $C_{jm}^+$ is, in general terms, a complex number. As we have the freedom of choice, we take $C_{jm}^+$ to be real and positive. The same argument applies to $C_{jm}^-$.}:
\begin{equation}
    C_{jm}^+ = \hbar \sqrt{j(j+1)-m(m+1)}
\end{equation}

Similarly, we can obtain:
\begin{equation}
    C_{jm}^- = \hbar \sqrt{j(j+1)-m(m-1)}
\end{equation}

So, we now have the raising and lowering operators completely defined:
\begin{definition}
    The raising and lowering operators $J_+$ and $J_-$ are given by:
    \begin{equation}
        J_\pm = J_x \pm iJ_y
    \end{equation}
    Their action on a ket $\ket{j,m}$ raises or lowers the second quantum number $m$ by one unit, respectively:
    \begin{equation}
        J_\pm\ket{j,m} = \hbar \sqrt{j(j+1)-m(m\pm1)}\ket{j,m\pm 1}
    \end{equation}
\end{definition}

\subsubsection{Matrix representation of angular momentum}

The formalism we discussed in the previous section is completely general and independent on the choice of representation. There are many ways to represent the angular momentum operators and their eigenstates, and in this section we will discuss the matrix representation of angular momentum, where eigenkets and operators are represented by column and square matrices, respectively.

Since $\vec{J}\,^2$ and $J_z$ commute, the set of their common eigenstates $\{\ket{j, m}\}$ can be chosen as a basis. This basis is discrete, orthonormal and complete. The orthonormality condition and the completeness relation can be expressed, respectively, as:
\begin{equation}
    \braket{j',m'|j,m} = \delta_{j',j}\delta_{m',m} \qquad\sum_{j}\sum_{m=-j}^{j}\ket{j,m}\bra{j,m} = \Imat
\end{equation}
The matrix elements of the $\vec{J}\,^2$ and $J_z$ operators in this basis are:
\begin{equation}
    \begin{split}
        \bra{j',m'}\vec{J}\,^2\ket{j,m} &= \hbar^2j(j+1)\delta_{j',j}\delta_{m',m} \\
        \bra{j',m'}J_z\ket{j,m} &= \hbar m\delta_{j',j}\delta_{m',m}
    \end{split}
\end{equation}
Clearly, $\vec{J}\,^2$ and $J_z$ are diagonal in this basis, their diagonal elements being $\hbar^2j(j+1)$ and $\hbar m$, respectively. However, as $J_\pm$ do not commute with $\vec{J}\,^2$ and $J_z$, they are not diagonal in this basis. Their matrix elements are:
\begin{equation}
    \bra{j',m'}J_\pm\ket{j,m} = \hbar \sqrt{j(j+1)-m(m\pm 1)}\delta_{j',j}\delta_{m',m\pm 1} \\
\end{equation}
The matrices of $J_x$ and $J_y$ are obtained by adding and subtracting the matrices of $J_+$ and $J_-$, respectively, as seen in \textbf{Equation \ref{eq:jx_and_jy_in_terms_of_jp_and_jm}}:
\begin{equation}
    J_x = \frac12(J_+ + J_-)\qquad J_y = \frac{1}{2i}(J_+ - J_-)
\end{equation}