\section*{Generalized Eigenvector}

When the eigenspaces of a matrix $A$ do not span the whole of the vector space where $A$ works in, which means that there does not exist a basis of eigenvectors of $A$, we can find generalized eigenvectors to complete the basis. Generalized eigenvectors $v$ are found by solving the equation $(A-\lambda_i\Imat)v=v_i$, where $v_i$ is an eigenvector of $A$.\\

Generalized eigenvectors $v$ of $A$ verify $v \in Ker(A-\lambda\Imat)^j$, for some $j\in \N$. In other words, $(A-\Imat)^{j+n}v=0$ for some $j$ and for any $n\in \N$. More graphically, they are vectors that fall into the eigenspace of $(A-\lambda\Imat)$ after having the transformation applied $j$ times on them.\\

Note that the fact that $Ker(A-\lambda\Imat)^k=Ker(A-\lambda\Imat)^{k+n},\;\forall n\in \N$, for some $k\in \N$ means that $(A-\lambda\Imat)$ is nilpotent.