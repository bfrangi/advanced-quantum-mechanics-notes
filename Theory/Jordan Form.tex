\section*{Jordan Form}

The Jordan form takes square matrices into box-diagonal or bi-diagonal form, which has the eigenvalues in the diagonal and some ones right above some of the eigenvalues (in those columns corresponding to generalized eigenvectors). For example:

$$J=\begin{pmatrix}\lambda_1&1&0&0&0\\0&\lambda_1&0&0&0\\0&0&\lambda_3&1&0\\0&0&0&\lambda_4&1\\0&0&0&0&\lambda_4\end{pmatrix}$$

This means that in this matrix we have one eigenvector $v_1$ with an associated generalized eigenvector $v_2$ and another eigenvector $v_3$ with two associated generalized eigenvectors $v_4$ and $v_5$.\\

Any square matrix $A$ is similar to a Jordan Form Matrix, $A=SJS^{-1}$, where $J$ is the Jordan Form and $S$ is a matrix that contains the eigenvectors/generalized eigenvectors in its columns (in the same order as the eigenvalues are ordered in $J$, and always placing each eigenvector immediately followed by each of the derived generalized eigenvectors, in order of how they were computed).\\

Two matrices are similar iff their Jordan forms are the same.\\