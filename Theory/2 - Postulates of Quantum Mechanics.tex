\section{Postulates of Quantum Mechanics}

\subsection{Introduction}

Quantum mechanics is based on a number of postulates, which themselves are based on a range of experimental observations. \textbf{These postulates cannot be derived}, and they result exclusively from experiment. They are the minimal set of assumptions that one needs in order to build the theory of quantum mechanics. \textbf{The validity of the postulates of quantum mechanics can only be determined inferentially}: the theory works extremely well, so the postulates must also be valid. This represents its experimental justification.

The first four postulates concern the state of a system at a give time, whereas the last postulate concerns the time evolution of the system.

\begin{postulate} \label{post_1}
    \textbf{State of a System:} The state of any physical system is specified, at each time $t$, by a state vector $\ket{\psi(t)}$ in a Hilbert space $\Hbt$; $\ket{\psi(t)}$ contains (and serves as the basis to extract) all the needed information about the system. Any superposition of state vectors is also a state vector.
\end{postulate}

\begin{postulate} \label{post_2}
    \textbf{Observables and Operators:} To every physical quantity $a$, called an observable or dynamical variable, there corresponds a linear Hermitian operator $A$ whose eigenvectors form a complete basis.
\end{postulate}

\begin{postulate} \label{post_3}
    \textbf{Measurements and Eigenvalues of Operators:} The measurement of an observable $a$ may be represented formally by the action of its operator $A$ on a state vector $\ket{\psi(t)}$. The only possible result of such a measurement is one of the eigenvalues $a_n$ (which are real) of the operator $A$. If the result of a measurement of $A$ on a state $\ket{\psi(t)}$ is $a_n$, the state of the system \textit{immediately after} the measurement changes to $\ket{\psi_n}$.
\end{postulate}

\begin{postulate} \label{post_4}
    \textbf{Probabilistic Outcome of Measurements:}
    \begin{itemize}
        \item \textbf{Discrete Spectra:} When measuring an observable $a$ of a system in a state $\ket{\psi}$, the probability of obtaining one of the non-degenerate eigenvalues $a_n$ of the corresponding operator $A$ is given by:
        \begin{equation}
            P_n(a_n) = \frac{|\braket{\psi_n|\psi}|^2}{\braket{\psi|\psi}} = |a_n|^2
        \end{equation}
        where $\ket{\psi_n}$ is the eigenstate of $A$ with the eigenvalue $a_n$ and we have assumed that the state is normalised. If the eigenvalue $a_n$ is $m$-degenerate, $P_n$ becomes:
        \begin{equation}
            P_n(a_n) = \frac{\sum_{j=1}^m|\braket{\psi_n^j|\psi}|^2}{\braket{\psi|\psi}} = \sum_{j=1}^m |a_n^{(j)}|^2
        \end{equation}
        where, again, we assume normalised states.
        
        The act of measurement changes the state of the system from $\ket{\psi}$ to $\ket{\psi_n}$. If the system is already in an eigenstate $\ket{\psi_n}$ of $A$, a measurement of $a$ yields with certainty the corresponding eigenvalue $a_n$: $A\ket{\psi_n} = a_n\ket{\psi_n}$.

        The mean value of an observable for a state $\ket{\psi}$ can be then calculated as:
        \begin{equation*}
            \braket{a}_\psi = \sum_n a_n P_n(a_n) = \sum_n a_n \sum_{j=1}^m |a_n^j|^2 = \sum_n a_n \sum_{j=1}^m \braket{\psi|\psi_n^j}\braket{\psi_n^j|\psi} = 
        \end{equation*}
        \begin{equation*}
            = \sum_n \sum_{j=1}^m \braket{\psi|a_n|\psi_n^j}\braket{\psi_n^j|\psi} = \sum_n \sum_{j=1}^m \braket{\psi|A|\psi_n^j}\braket{\psi_n^j|\psi} = \bra{\psi}A \left[\sum_n \sum_{j=1}^m \ket{\psi_n^j}\bra{\psi_n^j}\right]\ket{\psi} =
        \end{equation*}
        \begin{equation}
            = \bra{\psi}A\ket{\psi}
        \end{equation}

        \item \textbf{Continuous Spectra:} we can extend the previous relations to continuous spectra to determine the probability density that a measurement of $A$ yields a value between $a$ and $a + da$ on a system which is initially in a state $\ket{\psi}$:
        \begin{equation}
            \frac{dP(a)}{da} = \frac{|\braket{\psi_a|\psi}|^2}{\braket{\psi|\psi}} = |\psi(a)|^2
        \end{equation}
        where, again, we assume that the state $\ket{\psi}$ is normalised. 
        
        The mean value of an observable for a state $\ket{\psi}$ can be then calculated as:
        \begin{equation*}
            \braket{a}_\psi = \int a\, dP(a) = \int a|\psi(a)|^2\,da = \int a\,|\braket{\psi_a|\psi}|^2\,da = \int a\braket{\psi|\psi_a}\braket{\psi_a|\psi}\,da = 
        \end{equation*}
        \begin{equation*}
            = \int \bra{\psi}a\ket{\psi_a}\braket{\psi_a|\psi}\,da = \int \bra{\psi}A\ket{\psi_a}\braket{\psi_a|\psi}\,da = \bra{\psi}A\left(\int\ket{\psi_a}\bra{\psi_a}\,da\right)\ket{\psi}
        \end{equation*}
        \begin{equation}
            = \braket{\psi|A|\psi}
        \end{equation}

    \end{itemize}
\end{postulate}

\begin{postulate} \label{post_5}
    \textbf{Time Evolution of State Vectors:} The state vector $\ket{\psi(t)}$ of a system evolves in time according to the Schrödinger equation:
    \begin{equation}
        i\hbar\frac{\partial \ket{\psi(t)}}{\partial t} = H\ket{\psi(t)}
    \end{equation}
    where $H$ is the Hamiltonian operator corresponding to the total energy of the system.
\end{postulate}

\subsection{Observables and operators} \label{observables_and_operators}

An observable is a dynamic variable that can be measured. According to \textbf{Postulate \ref{post_2}}, a Hermitian operator is associated with every physical observable. In previous sections, we have seen that the position representation of the linear momentum operator is given by:
\begin{equation}
    \vec{P} = -i\hbar\vec{\nabla}
\end{equation}

In general, any function $f(\vec{r}, \vec{p})$ can be ``quantised'' (made into a function of operators) by replacing $\vec{r}$ and $\vec{p}$ with the position and momentum operators $\vec{R}$ and $\vec{P}$, respectively:
\begin{equation}
    f(\vec{r}, \vec{p}) \rightarrow F(\vec{R}, \vec{P})
\end{equation}

Some examples of operators are shown in \textbf{Table \ref{operators}}.

\begin{table}[htbp] \label{operators}
    \def\arraystretch{1.5}
    \centering
    \begin{tabular}{lr}
        \hline
        \textbf{Observable} & \textbf{Corresponding Operator} \\
        \hline
        $\vec{r}$ & $\vec{R}$ \\
        $\vec{p}$ & $\vec{P} = -\hbar \vec{\nabla}$ \\
        $T = \frac{p^2}{2m}$ & $T = -\frac{\hbar^2}{2m}\nabla^2$ \\
        $E = \frac{p^2}{2m} + V(\vec{r}, t)$ & $H = -\frac{\hbar^2}{2m}\nabla^2 + V(\vec{R}, t)$ \\
        $\vec{L} = \vec{r}\times\vec{p}$ & $\vec{L} = -i\hbar\vec{R}\times\vec{\nabla}$ \\
        \hline
    \end{tabular}
    \caption{Some observables and their corresponding operators.}
\end{table}

\subsubsection{Compatibility of observables}

Two observables $a$ and $b$ are said to be \textbf{compatible} if their corresponding operators $A$ and $B$ commute. This means that there exists a common set of eigenvectors for both operators. In other words, if $\ket{\psi_n}$ is an eigenvector of $A$ with eigenvalue $a_n$, then it is also an eigenvector of $B$ with eigenvalue $b_n$. It is said that such pair $A$ and $B$ of observables form a  \textbf{complete set of commuting observables} (CSCO).

Two observables can only be measured simultaneously if they are compatible, and this is a fact that is tightly connected to \textbf{Postulate \ref{post_3}}: when we measure an observable (we can think of this as applying its operator to the state), we obtain one of the eigenvalues of the operator, and immediately the system changes to the eigenstate corresponding to that eigenvalue. In order to be able to measure two different observables at the same time, the system must be able to change to an eigenstate that is common to \textit{both} operators. This is only possible if the operators commute, because for two operators to share the basis of eigenvectors they \textit{must} commute.

Furthermore, for commuting operators, it does not matter in what order we measure their corresponding observables. Measurement of one does not affect the measurement of the other. On the other hand, for non-commuting (incompatible) operators, measurement of one produces loss of information obtained from the measurement of the other.

\subsection{Conservative systems}

When the Hamiltonian of a system does not depend explicitly on time, the system is said to be \textbf{conservative}. In classical mechanics, the most important consequence of such a situation is the conservation of energy over time\footnote{Actually, eigenvectors \textit{can} change over time, but only by a phase.}. In other words, \textbf{the energy of the system is a constant of motion}. This also means that the eigenvalues and eigenvectors of the Hamiltonian operator are constant over time. Then, any state can be expanded in terms of the eigenvectors of the Hamiltonian operator:
\begin{equation}
    \ket{\psi(t)} = \sum_n c_n(t)\ket{\psi_n},\qquad c_n(t) = \braket{\psi_n|\psi(t)}
\end{equation}
In this case, the time evolution of the state vector $\ket{\psi(t)}$ is given by the time-dependent Schrödinger equation:

\begin{equation}
    i\hbar\frac{\partial \braket{\psi_n|\psi(t)}}{\partial t} = \braket{\psi_n|H|\psi(t)} \stackrel{\footnotemark}{\Longrightarrow}\footnotetext{Here, we use the eigenvalue equation: $\bra{\psi_n}H = E_n\bra{\psi_n}$ so that $\braket{\psi_n|H|\psi(t)} = E_n\braket{\psi_n|\psi(t)} =E_n c_n(t)$.} i\hbar\frac{\partial c_n(t)}{\partial t} = E_nc_n(t)
\end{equation}
which has the solution:
\begin{equation}
    c_n(t) = c_n(t_0)e^{-iE_n(t-t_0)/\hbar}
\end{equation}
Therefore:
\begin{equation} \label{time_evolution_cons}
    \ket{\psi(t)} = \sum_n c_n(t_0)e^{-iE_n(t-t_0)/\hbar}\ket{\psi_n}, \qquad c_n(t_0) = \braket{\psi_n|\psi(t_0)}
\end{equation}
and, for a continuous spectrum:
\begin{equation}
    \ket{\psi(t)} = \int c(E,t_0)e^{-iE(t-t_0)/\hbar}\ket{\psi_E}\,dE, \qquad c(E,t_0) = \braket{\psi_E|\psi(t_0)}
\end{equation}

\subsubsection{Stationary states}

A \textbf{stationary state} $\ket{\psi(t)}$ is a state that does not present any \textit{observable}\footnote{Meaning that any measurement of any observable always yields the same value.} change over time. They exist only in conservative systems, although not all states in a conservative system are stationary. For a stationary state, it is verified that the probability density function does not change over time\footnote{Remember that, in general, the shape of the probability density function can change, but the state \textit{must} remain normalised. It is in the particular case of stationary states that the shape is maintained also.}, so:

\begin{equation}
    \|\ket{\psi(t)}\|^2 = \|\ket{\psi(t_0)}\|^2 \Longrightarrow \ket{\psi(t)} = \ket{\psi(t_0)}e^{i\alpha(t)}
\end{equation}

for a certain real function $\alpha(t)$. Using the time-dependent Schrödinger equation:

\begin{equation}
    i \hbar \frac{\partial \ket{\psi(t)}}{\partial t} = H  \ket{\psi(t)}
\end{equation}

we obtain\footnote{Note that $\ket{\psi(t_0)}$ is a constant, and can be taken out of the derivative.}:

\begin{equation*}
    i\hbar \frac{\partial \ket{\psi(t)}}{\partial t} = i \hbar \frac{\partial \left(\ket{\psi(t_0)}e^{i\alpha(t)}\right)}{\partial t} = i\hbar \ket{\psi(t_0)} \frac{\partial (e^{i\alpha(t)})}{\partial t} = i\hbar \ket{\psi(t_0)} i \alpha'(t)e^{i\alpha(t)} = 
\end{equation*}
\begin{equation}
    = -\hbar \alpha'(t)\ket{\psi(t)} = H \ket{\psi(t)}
\end{equation}
Which means that $\ket{\psi(t)}$ is an eigenstate of the Hamiltonian operator with eigenvalue $-\hbar \alpha'(t)$. As the system is conservative, the eigenvectors evolve only by a phase and the eigenvalues of $H$ are constant, we have $\ket{\psi(t)}\to\ket{\psi_n}e^{i\alpha(t)}$ and $\alpha'(t)\to \alpha'$:

\begin{equation}
    -\hbar \alpha'\ket{\psi_n}e^{i\alpha(t)} = H \ket{\psi_n}e^{i\alpha(t)}
\end{equation}

It is now easy to see, looking at \textbf{Equation \ref{time_evolution_cons}}, that:
\begin{equation}
    \ket{\psi(t)} = \ket{\psi_n}e^{-iE_n(t-t_0)/\hbar}
\end{equation}

Therefore, we can define:

\begin{definition}
    Stationary states are eigenstates of the Hamiltonian of a conservative system. They evolve in time only by a phase, which is to say that they do not present any observable change over time.
\end{definition}

\subsubsection{Constants of motion}

A \textbf{constant of motion} $a$ is an observable that does not depend explicitly on time, and whose operator $A$ commutes with $H$:
\begin{equation}
    \frac{dA}{dt} = 0,\qquad [H,A] = 0
\end{equation}
Then, by \textbf{Theorem \ref{commuting_operator_base_thm}}, there is always a system of common eigenvectors for $H$ and $A$. These eigenvectors are stationary states, and their eigenvalues are called \textbf{good quantum numbers}.

\subsection{Superposition and interference}

\textbf{Superposition} in quantum mechanics does not work in the way we would expect from classical mechanics. Take the double slit experiment as an example. According to the classical superposition principle, the light intensity distribution on the screen is given by the sum of the intensities of the two waves coming from the two slits. In quantum mechanics, however, the light intensity distribution on the screen is proportional to the \textit{square} of the sum of the amplitudes of the two waves coming from the two slits:
\begin{equation}
    I(\vec{r}, t) \propto \left|\psi_1(\vec{r}, t) + \psi_2(\vec{r}, t)\right|^2 = \left|\psi_1(\vec{r}, t)\right|^2 + \left|\psi_2(\vec{r}, t)\right|^2 + \boxed{\psi_1^*(\vec{r}, t)\psi_2(\vec{r}, t) + \psi_1(\vec{r}, t)\psi_2^*(\vec{r}, t)}
\end{equation}

This extra term that appears as a result of the interference of the partial amplitudes is the origin of the \textbf{interference} phenomena in quantum mechanics.
