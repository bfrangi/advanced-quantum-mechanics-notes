\section{Postulates of Quantum Mechanics}

\subsection{Introduction}

Quantum mechanics is based on a number of postulates, which themselves are based on a range of experimental observations. \textbf{These postulates cannot be derived}, and they result exclusively from experiment. They are the minimal set of assumptions that one needs in order to build the theory of quantum mechanics. \textbf{The validity of the postulates of quantum mechanics can only be determined inferentially}: the theory works extremely well, so the postulates must also be valid. This represents its experimental justification.

The first four postulates concern the state of a system at a give time, whereas the last postulate concerns the time evolution of the system.

\begin{postulate}
    \textbf{State of a System:} The state of any physical system is specified, at each time $t$, by a state vector $\ket{\psi(t)}$ in a Hilbert space $\Hbt$; $\ket{\psi(t)}$ contains (and serves as the basis to extract) all the needed information about the system. Any superposition of state vectors is also a state vector.
\end{postulate}

\begin{postulate}
    \textbf{Observables and Operators:} To every physical quantity $a$, called an observable or dynamical variable, there corresponds a linear Hermitian operator $A$ whose eigenvectors form a complete basis.
\end{postulate}

\begin{postulate}
    \textbf{Measurements and Eigenvalues of Operators:} The measurement of an observable $a$ may be represented formally by the action of its operator $A$ on a state vector $\ket{\psi(t)}$. The only possible result of such a measurement is one of the eigenvalues $a_n$ (which are real) of the operator $A$. If the result of a measurement of $A$ on a state $\ket{\psi(t)}$ is $a_n$, the state of the system \textit{immediately after} the measurement changes to $\ket{\psi_n}$.
\end{postulate}

\begin{postulate}
    \textbf{Probabilistic Outcome of Measurements:}
    \begin{itemize}
        \item \textbf{Discrete Spectra:} When measuring an observable $a$ of a system in a state $\ket{\psi}$, the probability of obtaining one of the non-degenerate eigenvalues $a_n$ of the corresponding operator $A$ is given by:
        \begin{equation}
            P_n(a_n) = \frac{|\braket{\psi_n|\psi}|^2}{\braket{\psi|\psi}} = |a_n|^2
        \end{equation}
        where $\ket{\psi_n}$ is the eigenstate of $A$ with the eigenvalue $a_n$ and we have assumed that the state is normalised. If the eigenvalue $a_n$ is $m$-degenerate, $P_n$ becomes:
        \begin{equation}
            P_n(a_n) = \frac{\sum_{j=1}^m|\braket{\psi_n^j|\psi}|^2}{\braket{\psi|\psi}} = \sum_{j=1}^m |a_n^{(j)}|^2
        \end{equation}
        where, again, we assume normalised states.
        
        The act of measurement changes the state of the system from $\ket{\psi}$ to $\ket{\psi_n}$. If the system is already in an eigenstate $\ket{\psi_n}$ of $A$, a measurement of $a$ yields with certainty the corresponding eigenvalue $a_n$: $A\ket{\psi_n} = a_n\ket{\psi_n}$.

        The mean value of an observable for a state $\ket{\psi}$ can be then calculated as:
        \begin{equation*}
            \braket{a}_\psi = \sum_n a_n P_n(a_n) = \sum_n a_n \sum_{j=1}^m |a_n^j|^2 = \sum_n a_n \sum_{j=1}^m \braket{\psi|\psi_n^j}\braket{\psi_n^j|\psi} = 
        \end{equation*}
        \begin{equation*}
            = \sum_n \sum_{j=1}^m \braket{\psi|a_n|\psi_n^j}\braket{\psi_n^j|\psi} = \sum_n \sum_{j=1}^m \braket{\psi|A|\psi_n^j}\braket{\psi_n^j|\psi} = \bra{\psi}A \left[\sum_n \sum_{j=1}^m \ket{\psi_n^j}\bra{\psi_n^j}\right]\ket{\psi} =
        \end{equation*}
        \begin{equation}
            = \bra{\psi}A\ket{\psi}
        \end{equation}

        \item \textbf{Continuous Spectra:} we can extend the previous relations to continuous spectra to determine the probability density that a measurement of $A$ yields a value between $a$ and $a + da$ on a system which is initially in a state $\ket{\psi}$:
        \begin{equation}
            \frac{dP(a)}{da} = \frac{|\braket{\psi_a|\psi}|^2}{\braket{\psi|\psi}} = |\psi(a)|^2
        \end{equation}
        where, again, we assume that the state $\ket{\psi}$ is normalised. 
        
        The mean value of an observable for a state $\ket{\psi}$ can be then calculated as:
        \begin{equation*}
            \braket{a}_\psi = \int a\, dP(a) = \int a|\psi(a)|^2\,da = \int a\,|\braket{\psi_a|\psi}|^2\,da = \int a\braket{\psi|\psi_a}\braket{\psi_a|\psi}\,da = 
        \end{equation*}
        \begin{equation*}
            = \int \bra{\psi}a\ket{\psi_a}\braket{\psi_a|\psi}\,da = \int \bra{\psi}A\ket{\psi_a}\braket{\psi_a|\psi}\,da = \bra{\psi}A\left(\int\ket{\psi_a}\bra{\psi_a}\,da\right)\ket{\psi}
        \end{equation*}
        \begin{equation}
            = \braket{\psi|A|\psi}
        \end{equation}

    \end{itemize}
\end{postulate}

\begin{postulate}
    \textbf{Time Evolution of State Vectors:} The state vector $\ket{\psi(t)}$ of a system evolves in time according to the Schrödinger equation:
    \begin{equation}
        i\hbar\frac{\partial \ket{\psi(t)}}{\partial t} = H\ket{\psi(t)}
    \end{equation}
    where $H$ is the Hamiltonian operator corresponding to the total energy of the system.
\end{postulate}

