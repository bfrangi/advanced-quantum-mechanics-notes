\section*{Similarity/Diagonalization}

Two matrices $A$ and $B$ are similar iff $A=SBS^{-1}$ for some invertible $S$. Similar matrices have the same eigenvalues (not necessarily the same eigenvectors), the same determinant, the same rank, the same trace, the same characteristic polynomial, and the same Jordan form.\\

If there exists a basis of eigenvectors of $A$, then $A$ is similar to a diagonal matrix with the eigenvalues as its diagonal entries, $A=SDS^{-1}$. Here, the columns of $S$ are the eigenvectors of $A$, ordered in the same way as their respective eigenvalues are ordered in $D$. There only exists a basis of eigenvectors of $A$ if all $GM=AG$ for all eigenvalues of $A$, so this is the condition for $A$ to be diagonalized in this way.\\

A nilpotent matrix can never be diagonalized:\\

\textit{Proof:}

For a non-zero matrix $A$ such that $A^2=0$, we have, for some eigenvector $v\neq0$ and some eigenvalue $\lambda$ of $A$: 

$$Av=\lambda v \rightarrow A(Av)=A(\lambda v)\rightarrow A^2v=\lambda^2v \rightarrow 0=\lambda^2v\Leftrightarrow \lambda=0$$

This can clearly be extended to non-zero $A_{n \times n}$, where $A^k=0$ for some $k \in \N$. So, all eigenvalues of a nilpotent matrix are zero. Therefore, for a nilpotent matrix, $(A-\lambda\Imat)=A$, so the system we need to solve in order to find the eigenvectors is $Ax=0$. For there to exist a basis of eigenvectors of $A$, we would need $\text{dim}KerA=n$, so we need $A$ to have no pivots. However, this cannot happen unless the matrix is the zero matrix, so it will never happen here.\\