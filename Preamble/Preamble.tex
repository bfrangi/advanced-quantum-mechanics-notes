\usepackage[utf8]{inputenc}

%Colors:
\usepackage{xcolor}

%Math
\usepackage[makeroom]{cancel}%cancel lines math
\usepackage{dsfont}%fonts
\usepackage{amsmath,xparse}%matrices
\usepackage{amssymb}%math symbols
\usepackage{braket}
\newcommand{\s}{\text{ }}%Space in math mode
\newcommand{\trace}{\text{trace}\,}%Trace command
\newcommand{\rank}{\text{rank}\,}%Rank command
\newcommand{\R}{\mathbb{R}}%Real Numbers
\newcommand{\Hbt}{\mathcal{H}}%Hilbert Space
\newcommand{\F}{\mathcal{F}}%Hilbert Space
\newcommand{\C}{\mathbb{C}}%Complex Numbers
\newcommand{\N}{\mathbb{N}}%Natural Numbers
\newcommand{\Imat}{\mathds{1}}%Identity Matrix
% \newcommand{\ket}[1]{\left|#1\right\rangle}%Prints |#1⟩
% \newcommand{\bra}[1]{\left\langle#1\right|}%Prints ⟨#1|
% \newcommand{\braket}[2]{\langle#1|#2\rangle}%Prints ⟨#1|#2⟩
% \newcommand{\Braket}[2]{\left\langle#1\left|#2\right.\right\rangle}%Prints ⟨#1|#2⟩
% \newcommand{\braopket}[3]{\left\langle#1\left|#2\left|#3\right.\right.\right\rangle}%Prints ⟨#1|#2|#3⟩
\newcommand{\naturalset}[2]{#1=1,\;2,\;...,\;#2}%Prints #1= #1= 1, 2, ..., #2 in math mode
\newcommand{\vectorset}[3]{#1=\{#2_1,\;#2_2,\;...,\;#2_{#3}\}}%Prints #1= {#2_1,#2_2, ..., #2_{#3}} in math mode
\newcommand{\diagmatrix}[9]{%
\begin{pmatrix}
\lambda_1&1&0&0&0\\0&\lambda_1&0&0&0\\0&0&\lambda_3&1&0\\0&0&0&\lambda_4&1\\0&0&0&0&\lambda_4
\end{pmatrix}
}%Matrix with diagonal entries specified

% Setup matha font (mathabx.sty) for extracting specific symbols:
\DeclareFontFamily{U}{matha}{\hyphenchar\font45}
\DeclareFontShape{U}{matha}{m}{n}{
      <5> <6> <7> <8> <9> <10> gen * matha
      <10.95> matha10 <12> <14.4> <17.28> <20.74> <24.88> matha12
      }{}
\DeclareSymbolFont{matha}{U}{matha}{m}{n}
% Define a plus/minus character from that font (from mathabx.dcl):
\DeclareMathSymbol{\PM}{2}{matha}{"08}% to completely replace the \pm character, replace \PM with \pm
\ExplSyntaxOn

\NewDocumentCommand{\diagonal}{O{b}m}
 {% #1 = fences, #2 = entries, comma separated
  \egreg_diagonal:nn { #1 } { #2 }
 }

\seq_new:N \l__egreg_diagonal_entries_seq
\seq_new:N \l__egreg_diagonal_row_seq

\cs_new_protected:Nn \egreg_diagonal:nn
 {
  \seq_set_from_clist:Nn \l__egreg_diagonal_entries_seq { #2 }
  \begin{#1matrix}
  \int_step_function:nN { \seq_count:N \l__egreg_diagonal_entries_seq } \__egreg_diagonal:n
  \end{#1matrix}
 }

\cs_new_protected:Nn \__egreg_diagonal:n
 {% #1 = row number
  \seq_clear:N \l__egreg_diagonal_row_seq
  \int_step_inline:nn { \seq_count:N \l__egreg_diagonal_entries_seq }
   {
    \int_compare:nTF { #1 == ##1 }
     {
      \seq_put_right:Nx \l__egreg_diagonal_row_seq
       {
        \seq_item:Nn \l__egreg_diagonal_entries_seq { #1 }
       }
     }
     {
      \seq_put_right:Nn \l__egreg_diagonal_row_seq { 0 }
     }
   }
  \seq_use:Nn \l__egreg_diagonal_row_seq { & } \\
 }

\ExplSyntaxOff

%Page Margins:
\usepackage[a4paper, total={7in, 10in}]{geometry}

%paragraph Indent and Spacing
\setlength{\parindent}{0em}
\setlength{\parskip}{0.5em}

%Images:
\usepackage{graphicx}
\graphicspath{ {./images/} }
\usepackage{eso-pic}
\usepackage{wrapfig}%Wrapping Functionality
\usepackage{float}
\usepackage{chngcntr}% Count Figures as below v v v
\counterwithout{figure}{section}%          CONTINUOUS
\counterwithout{table}{section}%           CONTINUOUS

%Enumerate
\usepackage{enumitem} 

%Language
\usepackage[english]{babel}

%Environments
\usepackage{amsthm, amsfonts}
\usepackage{thmtools}
\usepackage{float}
\usepackage{framed}
\usepackage{tcolorbox}

\colorlet{LightGray}{gray!15}
\colorlet{LightOrange}{orange!15}
\colorlet{LightGreen}{green!15}
\colorlet{LightBlue}{blue!15}
\colorlet{LightRed}{red!15}

\newcommand{\HRule}[1]{\rule{\linewidth}{#1}}

\declaretheoremstyle[name=Theorem,]{thmsty}
\declaretheorem[style=thmsty,numberwithin=section]{theorem}
\tcolorboxenvironment{theorem}{colback=LightRed}

\declaretheoremstyle[name=Proposition,]{prosty}
\declaretheorem[style=prosty,numberlike=theorem]{proposition}
\tcolorboxenvironment{proposition}{colback=LightOrange}

\declaretheoremstyle[name=Definition,]{prcpsty}
\declaretheorem[style=prcpsty,numberlike=theorem]{definition}
\tcolorboxenvironment{definition}{colback=LightBlue}

%TOC: Fill Table of Contents with Dots
\usepackage{tocloft} %Fill ToC with dotted line
\renewcommand{\cftpartleader}{\cftdotfill{\cftdotsep}} % for parts
\renewcommand{\cftsecleader}{\cftdotfill{\cftdotsep}} % for sections

%Hyperlinks
\usepackage{hyperref}
